
\resumesection{Professional Experience}

\normalsize
\resumesubsection{Los Alamos National Laboratory}{Los Alamos, New Mexico}

\begin{minipage}{\textwidth}
     \begin{tabular}{r|ll}
	     2017 - \emph{Present} & \textbf{Theoretical Design Scientist}  & \textit{XTD-PRI, Primary Physics}  \\
	     2016 - 2017 & \textbf{Graduate Research Assistant}  & \textit{NEN-5, Systems Design \& Analysis} | 
									     \textit{ISR-1, Space Science \& Applications} \\
	     % 2016 - \emph{Present} & \textbf{Graduate Research Assistant} & \textit{ISR-1, Space Science \& Applications} \\
	     2011 - 2012 & \textbf{Post Master's Research Assistant}  & \textit{W-13, Advanced Engineering Analysis} \\
	     2010 & \textbf{Summer Intern} & \textit{XCP-3, Monte Carlo Codes} \\
     \end{tabular}
\end{minipage}

\begin{center}
	\begin{minipage}{0.99\textwidth}
		\begin{center}
			\textsc{\Large Performing tasks related to the development, design, implementation, testing, and validation of multiple computational physics codes and projects} \vspace{2mm}
		\end{center}
	\end{minipage}
\end{center}
\begin{minipage}{\textwidth}
	\begin{center}
		\begin{itemize}
			\item Writing software in Python, Fortran, C/C++, C\#, Matlab, Bash, HTML, and Javascript
			\item Unit-testing and building the framework for the Continuous Integration of the Common Modeling Framework (CMF)
			\item Adding features such as $\delta$-ray production, correlated secondary particles, and detector response functions to \textsc{MCNP6} 
			\item Creating tools for the automated analysis of a variety of detector systems, sources, and configurations for the Nuclear Detection Figure of Merit (NDFOM) project
	    		\item Multi-physics coupling of radiation transport in \textsc{MCNP6} and finite-element analysis in \textsc{Abaqus/CAE} % for the Engineering Campaign-7 Nuclear Survivability project % [Pubs: \ref{useful_prompt},\ref{mcnp_efforts}].
	    		\item Developing computational human phantoms for health and medical physics applications with \textsc{MCNP6} 
			\item Utilizing high performance computing systems for running advanced physics simulations and analysis
			\item Presenting at conferences and publishing articles in their proceedings [Pubs: \ref{ieee_nss_paper}, \ref{ans_summer2017_paper}, \ref{ans_summer2017_jrt_paper}, \ref{caari_2016_paper_1}, \ref{caari_2016_paper_2}, \ref{ans_winter2016_paper}, \ref{useful_prompt}, \ref{mcnp_efforts}, \ref{ans_2012_paper}, \ref{abaqus_2012_paper}]
			\item Performing additional software development tasks such as version control using Git \& Mercurial, configuring and securing websites through Apache, database management using SQL, and maintaining Linux servers
		\end{itemize}
	\end{center}
\end{minipage}

% Footnote for Clearance
%\begin{minipage}{\textwidth}
	%\begin{center}
		%%\footnotesize \textsuperscript{*} Paperwork submitted June, 2016 for reinstatement of DOE Q-level clearance.
	%\end{center}
%\end{minipage}

\resumesubsection{Medical College of Wisconsin}{Milwaukee, Wisconsin}

\begin{minipage}{\textwidth}
	\begin{tabular}{r|ll}
		2012 - 2016 & \textbf{Graduate Research Assistant} & \textit{Department of Biophysics}  \\
		% 2014 - 2016 & \textbf{Biophysics Representative, IT Liaison} & \textit{Graduate Student Council}  \\
			%\textbf{Information Technology Liason} & \textit{Graduate School of Biomedical Sciences} & 2015-2016 \\
	\end{tabular}
\end{minipage}

\begin{center}
	\begin{minipage}{0.9\textwidth}
		\begin{center}
			%\textsc{\Large Conducted research in Magnetic Resonance Imaging pusuing a Master's Thesis} \vspace{2mm}
			\textsc{\Large Conducted research in translational medicine and Magnetic Resonance Imaging while pursuing a Master's Degree in Biophysics} \vspace{2mm}
		\end{center}
	\end{minipage}
\end{center}

\begin{minipage}{\textwidth}
	% \begin{itemize}
	\begin{center}
		% \textsc{\large Conducted research and performed tasks towards pursuit a Master's Thesis such as:} \vspace{2mm}
		\begin{itemize}
			\item Contributing to a successful R21 National Institute of Health research grant that funded my research
			\item Patenting a segmented reconstruction technique for artifact reduction in Magnetic Resonance Imaging (MRI) [Pat: \ref{lqsm_patent}]
			\item Acquiring hands-on laboratory experience such as assembling MRI equipment and handling research animals
			\item Interacting with patients, working with technicians, and collaborating with medical doctors to conduct clinical research
			\item Analyzing and processing large imaging datasets seeking clinical applications of our imaging technique %MRI d searching for clinically viable clinical imaging for detecting clinical applications
			\item Writing publications and presenting findings at various international conferences [Pubs: \ref{ismrm_2016_paper}, \ref{cmrr_2015_paper}, \ref{ismrm_2014_paper}]
	\end{itemize}
	\end{center}
\end{minipage}

\resumesubsection{University of Wisconsin - Madison}{Madison, Wisconsin}

\begin{minipage}{\textwidth}
		\begin{tabular}{r|ll}
			2008 - 2011 & \textbf{Student Research Assistant} & \textit{Department of Medical Physics} \\
			2010 - 2011 & \textbf{Chapter President} & \textit{American Nuclear Society} \\
		\end{tabular}
\end{minipage}

\begin{center}
	\begin{minipage}{0.9\textwidth}
		\begin{center}
			%\textsc{\Large Attended Classes, performed Medical Physics research, and actively participated in our student chapter of American Nuclear Society} \vspace{2mm}
			\textsc{\Large Assisted in medical physics research, ran the student chapter of American Nuclear Society, and volunteered at public outreach events} \vspace{2mm}
		\end{center}
	\end{minipage}
\end{center}

\begin{minipage}{\textwidth}
	\begin{center}
	\begin{itemize}
		\item Researching methods for non-invasive Quality Assurance assessment of radioactive brachytherapy seeds
		\item Serving as the student chapter president of American Nuclear Society (ANS) and managing organizational duties
		\item Mentoring and teaching undergraduates, K-12 students, and others through a variety of ANS outreach events %such as through volunteering at various events, such as Science Olympiad, middle and high school science fairs, and local Boy Scout chapters on achieving their Nuclear Science Merit Badge.
	\end{itemize}
	\end{center}
\end{minipage}

%%% EXTRa jargOn

			%\item Conducted background research, provided preliminary results, and assisted in the writing of an R21 National Institute of Health (NIH) research grant to help fund my graduate research.
   			% \item Patented a segmented reconstruction technique for artifact reduction in Magnetic Resonance Imaging [Pat: \ref{lqsm_patent}].
			% \item Collaborated with a diverse group of professionals, including medical doctors and imaging technologists, to perform clinical research, meet deliverables, and submit the findings to various international conferences [Pubs: \ref{ismrm_2016_paper},\ref{cmrr_2015_paper},\ref{ismrm_2014_paper}].
			% \item Gained hands-on technical and experimental research skills by working in a laboratory and acquiring data while performing research on animals.
			% \item Facilitated communication between students, staff, and the university's Information Technology (IT) team as the graduate school IT liaison.
		% \end{itemize}

		% \item Facilitated communication between students, staff, and the university's Information Technology (IT) team as the graduate school IT liaison.
		%\item Collected my own data, provided the necessary care for our experimental animals, and expanded my laboratory and engineering skills by performing a variety of hands-on experimental and engineering tasks
		%\item Collected pre-clinical experimental data on imaging animals
		%\item Expanded experimental capabilities by my own data, caring for animals, and performing hands-on experiments
		%\item Interacted directly with clinical patients and subjects to collect patient data for clinical studies
		%\item Facilitated communication between students and staff in the graduate school with the university's Information Technology group
		%\item Communicated the concerns of the entire graduate student body with the university's Information Technology team
		%\item Provided technical leadership and guidance on Information Technology (IT) needs to all MCW Graduate students as the system Liason
		%\item Demonstrated an ability to interact with patients in a clinical setting in addition to handling animals for pre-clinical translation 
		%\item Collaborated with faculty, staff, and medical students to address and improve graduate school IT needs

			% \item Gained experience in applications involving radiation transport, nuclear engineering, and radiation detector systems


			% \item Published reports and presented on new \textsc{MCNP} features [Pubs: \ref{ieee_nss_paper},\ref{ans_summer2017_paper},\ref{ans_summer2017_jrt_paper},\ref{caari_2016_paper_1},\ref{caari_2016_paper_2},\ref{ans_winter2016_paper}]
	    			%\item Coupling radiation transport and finite-element analysis for multi-physics capability for the Engineering Campaign-7 Nuclear Survivability project [Pubs: \ref{useful_prompt},\ref{mcnp_efforts}].
			% \item Performed additional software development tasks including version control with Git \& Mercurial,  apache configuration for website deployment, 
			%\item Developed framework for the Continuous Integration testing of the Common Modeling Framework (CMF)
			% \item Designed, wrote, tested, benchmarked, and published new features in \textsc{MCNP6} [Pubs: \ref{ieee_nss_paper},\ref{ans_summer2017_paper},\ref{ans_summer2017_jrt_paper},\ref{caari_2016_paper_1},\ref{caari_2016_paper_2},\ref{ans_winter2016_paper}]
	    		%\item Implemented new features in \textsc{MCNP6} through writing code, developing benchmarks, publishing reports, and presenting the features at various conferences [Pubs: \ref{ieee_nss_paper},\ref{ans_summer2017_paper},\ref{ans_summer2017_jrt_paper},\ref{caari_2016_paper_1},\ref{caari_2016_paper_2},\ref{ans_winter2016_paper}].
			% \item Secured
			% \item Implemented features for the design, simulation, and analysis of radiation detector systems for the Nuclear Detection Figure of Merit (NDFOM) project.
	    		%\item Gained significant knowledge and experience in the design, modeling, simulation, and analysis of a variety of radiation detectors for the Nuclear Detection Figure of Merit (NDFOM) project.
	    		%\item Transitioned NDFOM from version 2.0 to 3.0 by modularizing and refactoring the backend Python code and through developing a cleaner, more intuitive user interface for the customer.
	    		%\item Transitioned NDFOM from version 2.0 to 3.0 by modularizing and refactoring the backend Python code and through developing a cleaner, more intuitive user interface for the customer.
			% \item Utilized version control methods with Git, Mercurial, and SVN for collaboration on software projects
	    		% \item Assisted in the development, testing, validation, and verification of the combined radiation transport and finite-element analysis multi-physics capability for the Engineering Campaign-7 Nuclear Survivability project [Pubs: \ref{useful_prompt},\ref{mcnp_efforts}].
			% \item Managed server system SQL
	    		% \item Developed unstructured mesh human phantoms for health physics applications with \textsc{MCNP6} [Pubs: \ref{ans_2012_paper},\ref{abaqus_2012_paper}].
	    		%\item \textsuperscript{*}Acquired DOE Q-level security clearance and assisted on the analysis of weapons systems.
	    		% \item Created a software visualization package for finite element geometries in MCNP simulations.
	    		% \item Utilized high performance computing (HPC) systems and utilities for advanced physics simulations and analysis.
	    		%\item Managed the deployed server of NDFOM, including SQL database
            		%\item Utilized SQL databases, managing servers, and utilizing version control using Mercurial
	    		%\item Modeling detector system responses using \textsc{MCNP}, Django, HTML, Javascript, and Python
            		%\item Utilized SQL databases, managing servers, and utilizing version control using Mercurial
